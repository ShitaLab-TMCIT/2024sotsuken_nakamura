% 卒業研究前刷テンプレート
% lualatex用

\RequirePackage{plautopatch}

\documentclass[a4paper,10pt]{ltjsarticle}
\usepackage{luatexja}
\usepackage{enumitem} % リストカスタマイズ用
\usepackage{geometry}
\usepackage{multicol}
\usepackage{amsmath}
\usepackage{titlesec}
\usepackage{flushend}
\usepackage{graphicx}
\usepackage{here}

\geometry{
top=20mm,
bottom=20mm,
left=20mm,
right=20mm
}
\setlength{\columnsep}{7.5mm}

\setlength{\baselineskip}{14pt}
\setlength{\parindent}{1\zw}

\titleformat{\section}{\large\bfseries}{\thesection}{1\zw}{}
\titlespacing{\section}{0pt}{*1}{*1}
\titleformat{\subsection}{\large\bfseries}{\thesubsection}{1\zw}{}
\titlespacing{\subsection}{0pt}{*1}{*1}
\titleformat{\subsubsection}{\large\bfseries}{\thesubsubsection}{1\zw}{}
\titlespacing{\subsubsection}{0pt}{*1}{*1}

\pagestyle{empty}

\SetLabelAlign{parright}{\parbox[t]{\labelwidth}{\raggedleft#1}}

\makeatletter
\def\@maketitle{
\begin{flushright}
{\large \@date}
\end{flushright}
\begin{center}
{\LARGE \@title \par}
\end{center}
\begin{flushright}
{\large \@author}
\end{flushright}
\par\vskip 1.5em
}
\makeatother

\title{\huge 直線中継伝送におけるアクセス制御方式の干渉時の評価\\
\Large Evaluation of access control schemes in linear relay transmission during interference.
}

\author{
T5-25 \:中村 優\\
指導教員 \: 設樂 勇
}

\date{}

\begin{document}
% タイトル部分は1カラムで表示
\twocolumn[
\maketitle
]

% ---------
% 本文開始
% ---------
\section{はじめに}
先行研究[1]にて自由空間での自律分散ドローンによる3次元メッシュネットワーク環境においてのオーバリーチの問題を解決するために、送信信号の届く中継局まで一度に中継するCTR(Cooperation Through Relay)方式が提案されている。本稿では、干渉や誤りが起きたときのCTR方式のスループット特性を従来方式と比較し評価する。

\section{従来方式の概要}
まず、従来方式について説明する。図1に従来方式の概要を示す。従来の中継伝送では1ホップずつ中継するが、自由空間で伝搬損失が少ないため送信信号が中継先のドローン(図1\#3)より遠くのドローン(図1\#4)に到達し干渉が起こる。そのため、従来の方式ではオーバリーチ干渉によってチャネル利用効率が低下する。また従来の方式で干渉が起き再送を行うとき、伝送レートを下げることでSNR (Signal to Noise Ratio)が低くても受信できるようにしているが、伝送レート低下に伴う送信時間や再送によるオーバヘッドが増加してしまう。

\section{CTR方式の概要}
図2に示すCTR方式は、送信信号の届く範囲の最終中継局(図1\#4)まで一度に信号を送信し、通信経路の中継局(図1\#3)もパケットを受信する。最終中継局がパケットの受信に失敗した場合は、直線経路の中継局\#3が\#4の代わりに次の中継局にパケットを中継する。そのため、オーバリーチ干渉の影響を減らすとともに中継ホップ数も減るので中継オーバヘッドも削減することができる。

\section{比較結果}
本稿では、CTR方式で誤りが生じたときの効果について評価する。中継の伝送距離は1000mとし、50m間隔で直線状に20台配置した。アンテナの送受信利得は0dBi、送信電力は10dBmとした。周波数は2.4GHz、伝送レートはIEEE 802.11gの18Mbpsとし、伝搬損失は自由空間伝搬損失とした。従来の方式では再送時の伝送レートを一つ下の12Mbpsとする。中継局を3台スルーした場合におけるスループット特性を評価した。このときパケット誤り率を3%と20%とし、1000回試行したときの従来の方式とCTR方式の平均のスループット特性を比較した。

図3に誤り率が3%、図4に誤り率が20%のCTR方式と従来の方式の平均スループット特性を示す。この結果から、いずれの条件でもCTR方式が従来の方式よりスループットが下回ることがないことがわかる。また、誤り率が3%と20%を比べると20%の方が従来の方式よりスループットが高い。このことから、誤る確率が高いほどCTR方式が従来の方式より高いスループットを得られるという結果が得られた。

% ---------
% 文献リスト
% ---------
\bibliography{arxiv} % bibファイルを指定 (例: arxiv.bib)
\bibliographystyle{junsrt}

\end{document}
