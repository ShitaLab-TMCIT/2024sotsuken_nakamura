% 卒業研究前刷テンプレート
% lualatex用

\RequirePackage{plautopatch}

\documentclass[a4paper,10pt]{ltjsarticle}
\usepackage{luatexja}
\usepackage{enumitem} % リストカスタマイズ用
\usepackage{geometry}
\usepackage{multicol}
\usepackage{amsmath}
\usepackage{titlesec}
\usepackage{flushend}
\usepackage{graphicx}
\usepackage{here}

\geometry{
top=20mm,
bottom=20mm,
left=20mm,
right=20mm
}
\setlength{\columnsep}{7.5mm}

\setlength{\baselineskip}{14pt}
\setlength{\parindent}{1\zw}

\titleformat{\section}{\large\bfseries}{\thesection}{1\zw}{}
\titlespacing{\section}{0pt}{*1}{*1}
\titleformat{\subsection}{\large\bfseries}{\thesubsection}{1\zw}{}
\titlespacing{\subsection}{0pt}{*1}{*1}
\titleformat{\subsubsection}{\large\bfseries}{\thesubsubsection}{1\zw}{}
\titlespacing{\subsubsection}{0pt}{*1}{*1}

\pagestyle{empty}

\SetLabelAlign{parright}{\parbox[t]{\labelwidth}{\raggedleft#1}}

\makeatletter
\def\@maketitle{
\begin{flushright}
{\large \@date}
\end{flushright}
\begin{center}
{\LARGE \@title \par}
\end{center}
\begin{flushright}
{\large \@author}
\end{flushright}
\par\vskip 1.5em
}
\makeatother

\title{\huge 直線中継伝送におけるアクセス制御方式の干渉時の評価\\
\Large Evaluation of access control schemes in linear relay transmission during interference.
}

\author{
T5-25 \:中村 優\\
指導教員 \: 設樂 勇
}

\date{}

\begin{document}
% タイトル部分は1カラムで表示
\twocolumn[
\maketitle
]

% ---------
% 本文開始
% ---------
\section{はじめに}
\begin{samepage}
先行研究[1]にて自由空間での自律分散ドローンによる3次元メッシュネットワーク環境においてのオーバリーチの問題を解決するために、送信信号の届く中継局まで一度に中継する CTR(Cooperation Through Relay)方式が提案されている。本稿では、干渉/誤りが起きたときのCTR方式のスループット特性を従来方式と比較し評価する。
\end{samepage}

\section{CTR方式の概要}
\begin{samepage}
図1にCTR方式の概要を示す。通常の中継伝送では1ホップずつ中継するため、オーバリーチ干渉によってチャネル利用効率が低下する。

CTR方式では、送信信号の届く範囲の最終中継局(図1の\#4)まで一度に信号を送信し、通信経路の中継局(図1の\#3)もパケットを受信する。最終中継局がパケットの受信に失敗した場合は、直線経路の中継局\#3が\#4の代わりに次の中継局にパケットを中継する。

図2にCTR方式のアクセス制御を示す。送信局はACK(Acknowledgement)の返信時間が記述されたパケットを送信する。パケットを受信した中継局は送信局に対してACK返信時間にACKを返信する。このとき、ACKを受信した経路上の中継局はACKの送信待ちをキャンセルする。

最終中継局(図2の\#4)がパケットを受信できない場合は\#3が送信局の\#1にACKを送信し、\#4の代わりに中継する。これにより、オーバリーチ干渉の影響を減らすとともに中継ホップ数も減るためオーバヘッドを削減できる。
\end{samepage}



% ---------
% 文献リスト
% ---------
\bibliography{arxiv} % bibファイルを指定 (例: arxiv.bib)
\bibliographystyle{junsrt}

\end{document}
